\documentclass[thesis.tex]{subfiles}

\begin{document}

\chapter{The context}

orientational dynamics of small, rigid particles suspended in fluid flows

\begin{itemize}
	\item fluid
	\item flows
	\item particles
	\item small
	\item orientational
\end{itemize}

This chapter tells you about all the fantastic things that this thesis is not about.

\section{Fluid flow}

Many physical systems around us are fluids. The air we breathe, the water we drink, the blood in our veins are all fluids. As a working definition we can think of a fluid as a system where the constituent molecules move around more or less freely. Sometimes they interact with each other and exchange some energy. The collisions give rise to what you perceive as friction. You know that syrup has more friction than water: if you pull a spoon through syrup, more of your energy is expended colliding molecules than if you were to pull the spoon through water. A measure of how often and how violently the molecules collide is the viscosity of a fluid, and we say that syrup has higher viscosity than water. Now, in many real situations there is something more than one fluid to consider. We may mix something into the fluid, for example a drop of oil into water. Then what happens depends on how the water molecules interact with the oil molecules. As you probably have experienced, the oil molecules like each other, and the water molecules, too. Therefore the oil concentrates into a drop where as many oil molecules as possible may be neighbours with other oil molecules.

But this is all a very qualitative, and you may rightly say naive, description. One could say that a fundamental problem of fluid physics is to figure out where all the different molecules go. From that we should deduce where the oil drop goes, and how fast, or if it perhaps breaks up, or maybe merges with another drop. However, making something useful out of this molecular picture is very difficult\footnote{
\begin{minipage}[l]{9cm}
However, modern computers are now allowing us to simulate surprisingly large numbers of molecules. If Youtube still exists at the time of this reading, I recommend a look at a video produced at the Lawrence Livermore National Laboratory showing the interface between two molten metals using exactly this approach. \url{http://www.youtube.com/watch?v=Wr7WbKODM2Q}
\end{minipage}\,\,
\begin{minipage}[l]{1.5cm}
\includegraphics[width=1.5cm]{figs/ninebillion_qr.png}
\end{minipage}
}.
Just consider that in one litre of water there are about $10^{25}$ molecules. In fact, we are not even particularly interested in the specific details of every molecule - we are interested in the macroscopic, observable world that is built up from all these molecules. Now, this thesis is not at all concerned with the detailed motion of molecules, but I still wanted to start with this picture because sometimes it becomes important to remember the origin of the macroscopic motion.\todo{connect forward to chap's about noise, fokker-planckery}.

The discipline studying the macroscopic properties and motion of fluids is called fluid dynamics. Some typical quantities studied there are the fluid velocity and pressure. We can think of the velocity at a certain position in the fluid as the average velocity of all the molecules at that point. It is more or less equal to the velocity a grain of pollen assumes if put at that point in the fluid. The pressure is the force per area an object in contact with the fluid experiences, due to the constant bombardment of molecules. Think for example of the forces in a bottle of soda. We now return to our example of the oil drop in water, and see that the switch from a molecular view to a fluid dynamical view opens a new problem: if we do not keep track of the molecules, we instead have to keep track of which points in space contain oil and which contain water. Separating the two materials, there is a boundary surface which can deform over time so that the oil drop changes shape. This sounds very complicated. Indeed, drop dynamics is a topic of its own, which this thesis does not intend to cover. Instead, this thesis concerns \emph{rigid particles}. A rigid body in physics is an object whose configuration can be described by one position (usually chosen to be the center-of-mass) and the rotation around said point. So let me exchange the oil drop in our experiment for a rigid particle. Make it a small metal particle instead, let's imagine a coin. 

 	


 






\end{document}
