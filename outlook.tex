\documentclass[thesis.tex]{subfiles}

\begin{document}

\chapter{Summary \& outlook}
Successful research needs good questions. And the more we know, the better questions we are able to ask.
In this thesis, I have presented main results of the past two and a half years work. But maybe the most important results are the new question we are able to pose? I briefly touched on the outlook for each project, but allow me to conclude with a more complete summary.

As mentioned in \Secref{experiment}, we are currently pursuing refined experiments using very nearly axisymmetric particles. We hope that their results will conclusively show that we observe the quasi-periodic orbits of triaxial particles.

However, there are several open theoretical questions concerning the dynamics of triaxial particles in simple shear flow. 
First, the motion on tori described in \Secref{experiment} tells us that there is a conserved quantity. But which quantity? In the case of axisymmetric particles, we know that the Jeffery orbit constant is preserved, but what is the correct set of variables in the triaxial case? Is there a relation to the well-known Standard Map \cite{ott2002, strogatz2000}? 

Second, The orientational distribution of triaxial particles subject to Brownian noise can be computed by methods similar to those described in Appendices~\ref{sec:fpesphere} and \ref{app:fpe_sphere} of this thesis \cite{favro1960,brenner1972,hubbard1972}. We now know that even minor asymmetries of the particle shape may have a strong effect on the rotational dynamics. The question is whether a small amount of noise dominates this effect. Otherwise, it is important to consider the particle asymmetry when computing the orientational distribution.

Third, the results for weakly inertial particles described in Paper~B may be extended to the case of triaxial particles. Numerical observations of inertial ellipsoids \cite{lundell2011} indicate that the stable position is when the minor particle axis aligns with the vorticity direction. This periodic solution will correspond to a single point in the surface-of-section.

For axisymmetric particles in simple shear flow, the effects of fluid inertia ($\rep>0$) stands out as an important topic. In Paper~B we performed linear stability analysis of the log-rolling and tumbling modes, as function of particle shape. Our result is valid for small values of $\st$, and $\rep =0$. As mentioned in \Secref{inertiaoutlook}, calculations by Subramanian \& Koch \cite{subramanian2005,subramanian2006} indicate that the result is qualitatively different when $\rep>0$, and in particular when $\rep=\st$, corresponding to the case of neutrally buoyant particles. We will attempt to adapt the reciprocal theorem of \cite{lovalenti1993,subramanian2005,subramanian2006} in order to extend the stability analysis to finite particle Reynolds numbers.

In Paper~C we ventured into the topic of non-spherical particles in random and turbulent flows. I will just mention two questions we are discussing. First, we believe that computing the relative angles of colliding non-spherical particles is important. The angle at which particles collide likely impacts the outcome of the collision. Second, sometimes the small Kubo-expansion described in \Secref{tumbling} does not work. For example, when we compute the particle correlation function $\langle \ve n(0)\cdot\ve n(t)\rangle$, we find that the perturbation expansion contains secular terms: terms diverging as $t\to\infty$. Clearly, the correlation function does not diverge in reality \cite{pumir2011}. The divergence is an artefact of the perturbation expansion. This raises the question of how to apply the methods of singular perturbation theory to the problem.

\end{document}
