\documentclass[thesis.tex]{subfiles}

\begin{document}

\chapter{Outlook}
Successful research needs good questions. And the more we know, the better questions we can ask.
In this thesis, I have presented the main results of the past year's work. But perhaps the most important results are the new questions we are able to pose? I have already touched on the outlook for each project, but here I to conclude with a more complete summary.

I have described an experiment where we observe aperiodic tumbling of rod-like particles (see \Secref{experiment}). We argue that the aperiodicity originates in the imperfect axisymmetry of the plastic particles used. The Jeffery theory predicts that if the observed particle is axisymmetric, we will observe periodic tumbling. We are currently pursuing refined experiments using very nearly axisymmetric particles. We expect that their results will conclusively show that we observe the quasi-periodic orbits of triaxial particles.
However, there are several open theoretical questions concerning the dynamics of triaxial particles in simple shear flow. 
First, the orientational motion of triaxial ellipsoids are confined to tori, as described in \Secref{experiment}. This fact tells us that there is a conserved quantity. But which quantity? In the case of axisymmetric particles, we know that the Jeffery orbit constant is preserved, but what is the correct set of variables in the triaxial case? We want to investigate whether there is a relation to the dynamical system known as ``the standard map'' \cite{ott2002}.

Second, in Paper~B (and Appendices~\ref{sec:fpesphere} and \ref{app:fpe_sphere} of this thesis) we explain how to compute orientational distributions on a sphere. The method described may be generalised to rotations of triaxial particles subject to Brownian noise \cite{favro1960,brenner1972,hubbard1972}. We now know that even minor asymmetries of the particle shape may have a strong effect on the rotational dynamics. The question is whether a small amount of noise dominates this effect. Otherwise, it is important to consider the particle asymmetry when computing the orientational distribution.

Third, the results for weakly inertial particles described in Paper~B may be extended to the case of triaxial particles. Numerical observations of inertial ellipsoids indicate the existence of a limiting orbit, such that the minor particle axis aligns with the vorticity direction \cite{lundell2011}. This periodic solution will correspond to a single point in the surface-of-section.

For axisymmetric particles in simple shear flow, the effect of fluid inertia ($\rep>0$) stands out as an important topic. In Paper~B we performed linear stability analysis of the log-rolling and tumbling modes, as function of particle shape. Our results are valid for small values of $\st$, and $\rep =0$. As mentioned in \Secref{inertiaoutlook}, calculations by \nopagebreak{Subramanian~\&~Koch~\cite{subramanian2005,subramanian2006}} indicate that the result is qualitatively different when $\rep>0$, and in particular when $\rep=\st$, corresponding to the case of neutrally buoyant particles. We will attempt to extend the stability analysis to finite particle Reynolds numbers by generalising the methods of \cite{subramanian2005,subramanian2006}. In particular, we believe that an adaptation of a ``reciprocal theorem'' \cite{lovalenti1993} will enable us to perform the stability analysis without requiring the explicit solution of the Navier-Stokes equations.

In Paper~C we ventured into the topic of non-spherical particles in random and turbulent flows. I will just mention two questions we are discussing at present. First, we believe that computing the relative angles of colliding non-spherical particles is important. The angle at which particles collide likely impacts the outcome of the collision. Second, sometimes the small Kubo-expansion described in \Secref{tumbling} does not work. For example, when we compute the particle correlation function $\langle \ve n(0)\cdot\ve n(t)\rangle$, we find that the perturbation expansion contains secular terms: terms diverging as $t\to\infty$. Clearly, the correlation function cannot diverge in reality. The divergence is an artefact of the perturbation expansion. This raises the question of how to apply the methods of singular perturbation theory to the problem.

One of the delights, and at the same time torments, of theoretical work, is that it is rarely possible to foresee where your questions lead. We are confident that some of the above questions will bear fruit, but exactly which, and how, only time will tell.
\end{document}
