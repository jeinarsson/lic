\documentclass[thesis.tex]{subfiles}

\begin{document}

The motion of small particles suspended in fluid flows is a fundamental research topic attracting interest in many branches of science, as well as in technical applications. In some cases it is the actual motion of the particles which is of interest. For example, in the atmospheric sciences the collisions and aggregation of small drops are important to the formation of rain. Similarly, in astronomy it is believed that the collisions of small dust grains lead eventually to the formation of planets in the accretion disk around a star. Another example is in marine biology, where the dynamics of small planktonic organisms swirled around by the ocean is fundamental in understanding their feeding and mating patterns \cite{guasto2012}. 

In other contexts the motion of the individual particle is of lesser interest, instead its effects on the suspending fluid is the topic of study. The properties of so-called complex fluids, meaning fluids with suspended particles, are studied in the field of rheology. For instance, the ketchup effect\footnote{The one where ketchup is stuck in the bottle, and nothing happens, and then suddenly all the ketchup pours out at once, only to become innocently solid again on the plate.} exists because of all the microscopic particles suspended in the liquid \cite{bayod2008}. On a more serious note, the similarly sudden onset of landslides in clay soils is related to the complex fluid of water and clay particles \cite{coussot2002}. A fundamental question in rheology is how to relate the microscopic motion of the suspended particles to the macroscopic behaviour of the complex fluid.

In many circumstances it is important that the particles are non-spherical, and how they are oriented. For instance, the ash clouds from volcanic eruptions play an important role in the radiation budget, and therefore the climate, of our planet \cite{mather2003}. The ash particles are non-spherical \cite{gasteiger2011}, and their shapes and orientations influence how light and energy is reflected from the volcanic aerosol \cite{dubovik2002}. 

All the examples above share a basis in the question of how a particle responds to a given flow, and how the flow in return responds to the particle. It turns out, though, that answers to seemingly simple questions can be unexpectedly elusive. The question \emph{what is the drag force on a perfect sphere travelling with a constant velocity through a still fluid bath?} is arguably still open. Sir Stokes wrote down the problem in 1851 \cite{stokes1851}, but even after a significant scientific effort, including the development of new mathematical methods, we have but rather crude approximations \cite{goldenfeld2007}. 



\section*{This thesis}

This thesis concerns the orientational dynamics of small, rigid particles suspended in fluid flows. 

\section*{Disposition}

\begin{itemize}
	\item Part I: Introduction, background, context
	\item Part II: Our work up to now
	\item Part III: Appendices with calculations
	\item Part IV: Research papers
\end{itemize}


\end{document}
