\documentclass[thesis.tex]{subfiles}

\begin{document}

\chapter{Fokker-Planck equation on the sphere}\label{sec:fpesphere}

The Fokker-Planck equation takes the familiar form in cartesian coordinates:
\begin{align}
	\partial_t P &= -\nabla \cdot [\ve v P] + \mathcal D \nabla^2 P.\eqnlab{diffusion}
\end{align}
This equation is usually called the diffusion equation, because for a particle advected by the \emph{drift} $\ve v$ and randomly kicked around with the \emph{diffusion constant} $\mathcal D$, \Eqnref{diffusion} governs the evolution of the probability $P(\ve x,t)$ of finding the particle at $\ve x$ at time $t$. As it is written, \Eqnref{diffusion} is valid for isotropic and homogenous diffusion, in other words, the random kicks do not depend on the position of the particle, and they are equally probable in all directions. 

When the particle does not move in cartesian space, but confined to the unit sphere there are two plausible ways of defining a Fokker-Planck equation. The first, and perhaps most straightforward, is to find out how to represent the operator $\nabla$ on the surface of the sphere, and use \Eqnref{diffusion} directly. The second way is to derive the Fokker-Planck equation from the microscopic equations of motion, as described for instance in the book by van Kampen \cite{kampen2007}. This approach has certain advantages. First, it allows for the case that the random kicks have non-standard statistics. For instance, the random kicks may be stronger in one direction than another. The random kicks may also depend on the current state of the system $\ve x$. Deriving the Fokker-Planck equation will automatically give the correct amendments to \Eqnref{diffusion}. Second, the calculation gives an explicit expression for the diffusion constant $\mathcal D$. In the case of thermal diffusion, there exists a fluctuation-dissipation theorem that relates $\mathcal D$ to the temperature and fluid drag. But what if the source of noise is different? For example, if the random kicks originate in random fluctuations of the flow gradients, what is $\mathcal D$?

In the case of isotropic and homogenous noise, the two approaches must clearly be equivalent. In this appendix I will derive the Fokker-Planck equation for a unit vector $\ve n$ subject to random rotations $\ve \omega$, expressed in cartesian coordinates. I will show that the equation so obtained is equivalent to the angular portion of the Laplacian in spherical coordinates. 

In the following $\nabla$ represents the usual gradient in cartesian space, and $\partial_{\ve n}$ denotes the surface gradient confined to the sphere. Then in the end, $\partial_{\ve n}$ and $\partial^2_{\ve n}$ will be replace $\nabla$ and $\nabla^2$ in \Eqnref{diffusion} to give the correct Fokker-Planck equation on the unit sphere.

\section{The standard way}

 The surface gradient on a sphere is defined as the projection of $\nabla$ onto the surface $|\ve n|=1$:
\begin{align}
	\partial_{\ve n} &= (\ma I - \ve n \ve n \transpose) \nabla.\eqnlab{spherenabla1}
\end{align}
The interpretation is straightforward. Take the gradient $\nabla f$, which is a vector, and subtract the component in the outwards direction from the sphere, $-\ve n(\ve n \cdot \nabla f)$. The result is the portion of the gradient vector tangent to the sphere at $\ve n$.

In the following derivations I will often use index notation with implicit summation over repeated indices, in which \Eqnref{spherenabla1} reads
\begin{align*}
	(\partial_{\ve n})_i &= (\delta_{ij} - n_in_j)\partial_j.
\end{align*}
Here $\partial_j \equiv \partial/\partial n_j$, and $\delta_{ij}$ is the Kronecker symbol. We proceed and also compute $\partial^2_{\ve n}$ on component form:
\begin{align}
	\partial^2_{\ve n} &= (\partial_{\ve n})_i(\partial_{\ve n})_i \nn \\
	&= (\delta_{ij} - n_in_j)\partial_j(\delta_{ik} - n_in_k)\partial_k \nn \\
	&= \underbrace{\delta_{ij}\partial_j \delta_{ik} \partial_k}_{\partial_i\partial_i} - \underbrace{\delta_{ij}\partial_jn_in_k\partial_k}_{A} - \underbrace{n_in_j\partial_j\delta_{ik}\partial_k}_{-n_in_j\partial_i\partial_j} + \underbrace{n_in_j\partial_jn_in_k\partial_k}_{B}\eqnlab{partialn2start}
\end{align}
In order to compute $A$ and $B$ we evaluate product derivatives according to
\begin{align*}
	\partial_i f g &= \partial_i [f] g + f \partial_i[g] + fg \partial_i, \\
\end{align*}
where the last term is kept because we are manipulating an operator. Further, we use that
\begin{align*}
		\partial_i n_j &= \delta_{ij},\\
		n_in_i &= 1,\\
		\delta_{ii} &= 3,
\end{align*}
where the number $3$ is the number of spatial dimensions. Continuing, 
\begin{align*}
A &= -\delta_{ij}\partial_jn_in_k\partial_k \\
&= -\partial_in_in_k\partial_k \\
&= -(\delta_{ii}n_k\partial_k + n_i \delta_{ik} \partial_k + n_in_k\partial_i\partial_k)\\
&= -(3n_k\partial_k + n_i \partial_i + n_in_k\partial_i\partial_k)\\
%
B&=n_in_j\partial_jn_in_k\partial_k \\
	&= n_in_j\delta_{ij}n_k\partial_k + n_in_jn_i\delta_{jk}\partial_k + n_in_jn_in_k\partial_j\partial_k\\
	&=n_k\partial_k + n_k\partial_k + n_jn_k\partial_j\partial_k \\
	&= 2n_k\partial_k + n_jn_k\partial_j\partial_k
\end{align*}
Upon inserting $A$ and $B$ into \Eqnref{partialn2start} we obtain
\begin{align}
	\partial^2_{\ve n} &= \partial_i\partial_i - n_in_j\partial_i\partial_j - 2n_i\partial_i. \eqnlab{partialn2}
\end{align}
This is the result to which we will compare the result of our subsequent derivations.
\subsection{Relation to Laplace operator in spherical coordinates}
I now show that $\partial^2_{\ve n}$ corresponds to the angular part of the Laplace operator, as written in spherical coordinates $(r, \theta, \varphi)$:
\begin{align*}
	\nabla^2  &= \frac{1}{r^2}\partial_rr^2\partial_r  + \frac{1}{r^2 \sin \theta}\partial_\theta \sin\theta\partial_\theta  + \frac{1}{r^2\sin^2\theta}\partial^2_\varphi.  \\
	\intertext{Expand derivatives in $r$ and put $r=1$ to obtain}
	\nabla^2  &= \partial^2_r + 2\partial_r  + \frac{1}{\sin \theta}\partial_\theta \sin\theta\partial_\theta  + \frac{1}{\sin^2\theta}\partial^2_\varphi.  \\
\end{align*}
Now we use this relation to express $\partial^2_{\ve n}$ in spherical coordinates:
\begin{align*}
		\partial^2_{\ve n} &= \nabla ^2 - n_in_j\partial_i\partial_j - 2n_i\partial_i \\
		&= \nabla^2 - \left((\ve n \cdot \nabla)^2-\ve n\cdot\nabla\right) - 2\ve n\cdot\nabla \\
		&= \nabla^2 - (\ve n \cdot \nabla)^2 - \ve n\cdot\nabla \\
		\intertext{We know that $\ve n\cdot\nabla = r\partial_r$ in spherical coordinates:}
		&= \nabla^2 - (r\partial_r)^2 - r\partial_r \\
		\intertext{And like above, expand derivatives in $r$ and put $r=1$ to obtain}
		&= \nabla^2 - \partial^2_r - 2\partial_r \\
		&= \frac{1}{\sin \theta}\partial_\theta \sin\theta\partial_\theta  + \frac{1}{\sin^2\theta}\partial^2_\varphi. && \blacksquare
\end{align*}

\subsection{Relation to angular momentum operators}
Another common way of writing down differential operators on the surface of a sphere is to use the orbital angular momentum operator $\hat{\ve L}$ \cite{jackson1999}:
\begin{align}
	\hat{\ve L} &= -i \ve n \cross \nabla, \nn
	\intertext{from which it follows}
	-i \ve n \cross \hat{\ve L} &=  (\ma I - \ve n \ve n \transpose) \nabla = \partial_{\ve n},\eqnlab{ncrossl}\\
	-\hat{\ve L}^2 &= (\ma I - \ve n \ve n \transpose) \nabla \cdot (\ma I - \ve n \ve n \transpose) \nabla = \partial^2_{\ve n}.\eqnlab{lsquare}
\end{align}
There are several variants in literature, for example Doi \& Edwards \cite{doi1986} defines the rotation operator  $\mathcal R = i \hat{\ve L}$. Although the prefactors change, the idea is always the same. The relations \Eqnref{ncrossl} and \Eqnref{lsquare} are straightforward to prove by calculation. 

\proofheader{Proof of \Eqnref{ncrossl}}
\begin{align*}
	-i(\ve n \cross \hat{\ve L})_i &= i^2 \lc_{ijk}n_j\lc_{kpq}n_p\partial_q \\
	&= -(\delta_{ip}\delta_{jq}-\delta_{iq}\delta_{jp})n_jn_p\partial_q \\
	&= -(n_qn_i\partial_q - n_pn_p\partial_i) \\%&& \textrm{(since $n_pn_p=1$)}\\
	&= (\delta_{iq} - n_in_q)\partial_q && \blacksquare
\end{align*}

\proofheader{Proof of \Eqnref{lsquare}}
\begin{align*}
	-\hat L_i\hat L_i &= \lc_{ijk}n_j\partial_k\lc_{ipq}n_p\partial_q \\
	&= (\delta_{jp}\delta_{kq}-\delta_{jq}\delta_{kp})n_j\partial_kn_p\partial_q \\
	&= n_j\partial_kn_j\partial_k - n_j\partial_kn_k\partial_j \\
	&= \partial_k\partial_k + n_j \partial_j - n_j n_k \partial_j \partial_k - 3 n_j \partial_j\\
	&= \partial_k \partial_k - n_i n_j\partial_i\partial_j-2n_i\partial_i && \blacksquare
\end{align*}

\section{Derivation from equations of motion}
We now turn to the derivation of the Fokker-Planck equation from the equation of motion of a single particle. The general procedure is described in detail in the book by van Kampen \cite{kampen2007}, here I simply state the recipe: for a dynamical system of several variables we compute the drift for each variable $n_i$ over a short time interval $\delta t$ as
\begin{align}
	a^{(1)}_i &= \lim_{\delta t\to0}\frac{\langle \delta n_i\rangle}{\delta t},\eqnlab{a1def}
\end{align}
and the diffusion covariance matrix between all combinations of variables:
\begin{align}
	a^{(2)}_{ij} &= \lim_{\delta t\to0}\frac{\langle (\delta n_i)(\delta n_j)\rangle}{\delta t}.\eqnlab{a2def}
\end{align}
The resulting Fokker-Planck equation for $P(\ve n, t)$ is
\begin{align*}
	\frac{\partial P}{\partial t} &= -\frac{\partial}{\partial n_i}\left[a^{(1)}_i P\right] + \frac{1}{2}\frac{\partial^2}{\partial n_i \partial n_j}\left[a^{(2)}_{ij} P\right].
\end{align*}

\subsection{Formulas for the displacements $\delta \ve n$}
We start with the equation of motion for the vector $\ve n(t)$. Its time evolution is governed by 
\begin{align*}
	\dot {\ve n} &= \ve f(\ve n(t), t).
\end{align*}
In general, the displacement $\delta \ve n$ after a small time $\delta t$ is
\begin{align}
	\delta \ve n &= \ve n(\delta t) - \ve n_0 \nn\\
	&= \int_0^{\delta t}\!\!\!\!\!\rd t_1 \ve f (\ve n(t_1 ), t_1)\nn\\
	&= \int_0^{\delta t}\!\!\!\!\!\rd t_1 \ve f\left(\ve n_0 + \int_0^{t_1}\rd t_2 \ve f(\ve n(t_2),t_2), t_1\right)\nn\\
	&= \int_0^{\delta t}\!\!\!\!\!\rd t_1 \ve f (\ve n_0, t_1)+\int_0^{\delta t}\!\!\!\!\!\rd t_1 \ma F (\ve n_0, t_1)\int_0^{t_1}\rd t_2 \ve f(\ve n_0, t_2) + ...\eqnlab{deltan}
\end{align}
where $\ma F$ denotes the matrix of derivates
\begin{align*}
	\ma F(\ve n, t) &= \frac{\rd \ve f}{\rd \ve n}.
\end{align*}
The function $\ve f(\ve n, t)$ may in general include both deterministic terms and terms including Gaussian white noise. The deterministic part of $\ve f$ results in the \emph{drift} $\ve v$ in \Eqnref{diffusion}. In the following we will assume that $\ve f$ contains no deterministic terms in order to focus on the diffusion operator.

\subsection{Random angular velocities}
For a particle orientation $\ve n$ driven by only random angular velocities
\begin{align}
	\ve f(\ve n, t) &= \ve \omega(t) \cross \ve n, \qquad \begin{array}{ll}
  	 \langle\omega_i(t)\rangle &= 0, \\
  	 \langle\omega_i(0)\omega_j(t)\rangle &= 2\mathcal D \delta_{ij}\delta(t),
  \end{array}\eqnlab{eqnofmotion1}
\end{align}
where I anticipate the diffusion constant $\mathcal D$. For mathematical convenience, rewrite the cross product in terms of an antisymmetric matrix $\ma O$, such that $\ma O\ve n=\ve\omega \cross \ve n$, meaning
\begin{align*}
	\ve f(\ve n, t) &= \ma O(t) \ve n, \\
	\ma F &= \ma O,\\
	\intertext{where $\ma O$ is defined by}
	O_{ij} &= -\lc_{ijp}\omega_p, \intertext{and has statistics}
	\left\langle O_{ij}(t) \right\rangle &= 0,\\
	\left\langle O_{ij}(0)O_{kl}(t)\right\rangle &= \left\langle\lc_{ijp}\omega_p(0)\lc_{klq}\omega_q(t)\right\rangle \\
	&= \lc_{ijp}\lc_{klq}\left\langle \omega_p(0)\omega_q(t)\right\rangle \\
	&= 2\mathcal D\lc_{ijp}\lc_{klq} \delta_{pq}\delta(t)\\
	&= 2 \mathcal D(\delta_{ik}\delta_{jl}-\delta_{il}\delta_{jk})\delta(t).
\end{align*}
By insertion of  $\ve f$ and $\ma F$ into \Eqnref{deltan} we may now compute the average displacement. In order to ease the notation I call the initial condition $\ve n_0$ simply $\ve n$.
\begin{align*}
	\langle \delta \ve n\rangle &= \int_0^{\delta t}\!\!\!\!\!\rd t_1 \left\langle\ma O(t_1)\right\rangle\ve n+\int_0^{\delta t}\!\!\!\!\!\rd t_1 \int_0^{t_1}\rd t_2 \left\langle \ma O (t_1)\ma O(t_2)\right\rangle\ve n + \mathcal O(\delta t^2)\\
	&= \mathcal D\int_0^{\delta t}\!\!\!\!\!\rd t_1 (\ma I-3\ma I)\ve n  + \mathcal O(\delta t^2) \\
	&= -2\mathcal D \delta t \ve n  + \mathcal O(\delta t^2).
\end{align*}
Similarly we obtain the second moments of the displacements $\delta \ve n$ up to $\mathcal O(\delta t)$:
\begin{align}
	\langle \delta \ve n\delta \ve n\transpose\rangle &= \int_0^{\delta t}\!\!\!\!\!\rd t_1 \int_0^{\delta t}\!\!\!\!\!\rd t_2 \left\langle\ve f(\ve n,t_1) \ve f\transpose(\ve n, t_2)\right\rangle + \mathcal O(\delta t^2)\nn\\
	&= \int_0^{\delta t}\!\!\!\!\!\rd t_1 \int_0^{\delta t}\!\!\!\!\!\rd t_2 \left\langle \ma O(t_1)\ve n\ve n\transpose \ma O(t_2)\transpose \right\rangle + \mathcal O(\delta t^2).\nn\\
	\intertext{Insert the statistics for $\ma O$ to obtain}
	&= 2\mathcal D \int_0^{\delta t}\!\!\!\!\!\rd t_1 \int_0^{\delta t}\!\!\!\!\!\rd t_2 \delta(t_1-t_2)(\ma I|\ve n|^2-\ve n\ve n\transpose) + \mathcal O(\delta t^2)\nn\\
	&= 2\mathcal D \delta t (\ma I|\ve n|^2-\ve n\ve n\transpose) + \mathcal O(\delta t^2).\eqnlab{a2vec}
\end{align}
At this point we must refrain from using the fact that, eventually, $|\ve n|=1$. The equation of motion \eqnref{eqnofmotion1} only implies that $|\ve n|$ is conserved, not at which value. As a direct consequence the mean square displacements \eqnref{a2vec} ensure that the random jumps also conserve $|\ve n|$. If we were to change this fact by letting $|\ve n|=1$, while clearly the term $-\ve n\ve n\transpose$ is proportional to  $|\ve n|^2$, the random jumps will no longer conserve $|\ve n|$. The resulting Fokker-Planck equation would contain terms describing this erronous flux normal to the unit sphere\footnote{In fact, using $|\ve n|=1$ at this point leads to the Fokker-Planck equation $\partial_t P = \nabla\cdot(\ma I-\ve n\ve n\transpose)\nabla P$.}. Now, however, we conclude the derivation of the correct Fokker-Planck equation.

By the definitions in Eqns. \eqnref{a1def} and \eqnref{a2def}, the coefficients $a^{(1)}$ and $a^{(2)}$ are
\begin{align}
	a_i^{(1)} &= \lim_{\delta t\to0}\left\langle\frac{\delta n_i}{\delta t}\right\rangle = -2\mathcal D n_i,\nn\\
	a_{ij}^{(2)} &= \lim_{\delta t\to0}\left\langle\frac{\delta n_i\delta  n_j}{\delta t}\right\rangle = 2\mathcal D(\delta_{ij}n_kn_k - n_i n_j).\eqnlab{a1a2}
\end{align}
The Fokker-Planck equation for the orientational distribution is then
\begin{align}
	\frac{\partial P}{\partial t} &= -\partial_i\left[a^{(1)}_i P\right] + \frac{1}{2}\partial_i\partial_j\left[a^{(2)}_{ij} P\right]\nn\\
&= 2\mathcal D\partial_i \left[n_i P\right] + \mathcal D\partial_i \partial_j\left[(\delta_{ij}n_kn_k-n_in_j) P\right].\eqnlab{fpe1}	
\end{align}
This equation should correspond to the diffusion equation \Eqnref{diffusion}, with the drift $\ve v=0$ and the spherical surface Laplacian \Eqnref{partialn2} substituted for $\nabla^2$. In order to make the comparison to $\partial^2_{\ve n}$ in \Eqnref{partialn2}, we remove $\mathcal D$ and expand the product derivatives of the operator in the right hand side of \Eqnref{fpe1}.
\begin{align*}
	& 2\partial_i n_i + \partial_i \partial_j\left(\delta_{ij}n_kn_k-n_in_j\right)\\
	&= 2\delta_{ii} + 2n_i\partial_i + \partial_i\left[2\delta_{ik}n_k+n_kn_k\partial_i\right]  \\
	&\qquad - \partial_i\left[\delta_{ij}n_j + n_i\delta_{jj}+ n_in_j\partial_j\right]\\
	&=6 + 2n_i\partial_i + 2\delta_{kk} + 2n_i\partial_i + 2\delta_{ik}n_k\partial_i + n_kn_k\partial_i\partial_i \\
	&\qquad -\delta_{jj}-n_j\partial_j-3\delta_{ii}-3n_i\partial_i\\
	&\qquad-\delta_{ii}n_j\partial_j-n_i\delta_{ij}\partial_j-n_in_j\partial_i\partial_j \\
	\intertext{Here we may use $|\ve n|=1$, when it occurs outside derivatives:}
	&=6 + 2n_i\partial_i + 6 + 2n_i\partial_i + 2n_i\partial_i + \partial_i\partial_i \\
	&\qquad -3-n_j\partial_j-9-3n_i\partial_i\\
	&\qquad-3n_j\partial_j-n_i\partial_i-n_in_j\partial_i\partial_j \\
	&= \partial_i\partial_i -n_in_j\partial_i\partial_j - 2 n_i \partial_i = \partial^2_{\ve n}.
\end{align*}
This concludes the derivation of the Fokker-Planck equation for a vector $\ve n$ undergoing random rotations.

\subsection{Remark on drift terms}

In the above we neglected the deterministic terms of $\ve f(\ve n, t)$, in order to derive the correct diffusion operator. If we allow for a deterministic term in \Eqnref{eqnofmotion1}, the equation of motion becomes
\begin{align}
	\ve f(\ve n, t) &= \ve \Omega \cross \ve n + \ve \omega(t) \cross \ve n, \qquad \begin{array}{ll}
  	 \langle\omega_i(t)\rangle &= 0, \\
  	 \langle\omega_i(0)\omega_j(t)\rangle &= 2\mathcal D \delta_{ij}\delta(t).
  \end{array}\eqnlab{eqnofmotion2}
\end{align}
The deterministic term transfers directly to the average displacement
\begin{align*}
	\langle \delta \ve n\rangle &= (\ve \Omega \cross \ve n -2\mathcal D  \ve n)\delta t  + \mathcal O(\delta t^2).
\end{align*}
The corresponding Fokker-Planck equation thus has a drift term
\begin{align}
	\partial_t P &= -\nabla \cdot \left(\ve \Omega \cross \ve n P\right) + \partial^2_{\ve n}P.\eqnlab{fpe3}
\end{align}
Now, this may seem to contradict what I stated above: the Fokker-Planck equation on the sphere is \Eqnref{diffusion} with $\partial_{\ve n}$ substituted for $\nabla$. But the contradiction is only apparent. We may replace $\nabla$ with $\partial_{\ve n}$ in \Eqnref{fpe3} without any change, because
\begin{align*}
	(\nabla - \partial_{\ve n})\cdot[\ve \Omega\cross\ve nP] &= n_i n_j \partial_j \lc_{ikl}\Omega_kn_lP\\
	&=n_i n_j \lc_{ikl}[\partial_j \Omega_k] n_lP + n_i n_j \lc_{ikl}\Omega_k \delta_{jl} P \\
	&\qquad + n_i n_j \lc_{ikl}\Omega_k n_l \partial_j P   \\
	&= 0.
\end{align*}
Therefore \Eqnref{fpe3} may be written as
\begin{align}
	\partial_t P &= -\partial_{\ve n}\cdot \left(\ve \Omega \cross \ve n P\right) + \partial^2_{\ve n}P,\eqnlab{fpe4}
\intertext{or, by relations \Eqnref{ncrossl} and \Eqnref{lsquare}}	
	&= i\ve n\cross \hat{\ve L}\left(\ve \Omega \cross \ve n P\right) - \hat{\ve L}^2 P.
\end{align}

% \subsection{erronous version}

% By the definition in Eqns. \eqnref{a1def} and \eqnref{a2def}, the coefficients $a^{(1)}$ and $a^{(2)}$ are
% \begin{align*}
% 	a_i^{(1)} &= \lim_{\delta t\to0}\left\langle\frac{\delta n_i}{\delta t}\right\rangle = -2\mathcal D n_i,\\
% 	a_{ij}^{(2)} &= \lim_{\delta t\to0}\left\langle\frac{\delta n_i\delta  n_j}{\delta t}\right\rangle = 2\mathcal D(\delta_{ij} - n_i n_j).
% \end{align*}


% We must first derive the correct Fokker-Planck equation, and then restrict 
% The Fokker-Planck equation for the orientational distribution is then
% \begin{align}
% 	\frac{\partial P}{\partial t} &= -\partial_i\left[a^{(1)}_i P\right] + \frac{1}{2}\partial_i\partial_j\left[a^{(2)}_{ij} P\right]\\
% &= 2\mathcal D\partial_i \left[n_i P\right] + \mathcal D\partial_i \partial_j\left[(\delta_{ij}-n_in_j) P\right].\eqnlab{fpe2}	
% \end{align}
% This equation should correspond to the diffusion equation \Eqnref{diffusion}, with the drift $\ve v=0$ and the spherical surface Laplacian \Eqnref{partialn2} substituted for $\nabla^2$. In order to make the comparison to $\partial^2_{\ve n}$, we remove $\mathcal D$ and expand the product derivatives of the operator in the right hand side of \Eqnref{fpe2}:
% \begin{align*}
% 	& 2\partial_i n_i + \partial_i \partial_j\left(\delta_{ij}-n_in_j\right)\\
% 	&= 2\delta_{ii} + 2n_i\partial_i + \partial_i\partial_i - \partial_i\left[\delta_{ij}n_j + n_i\delta_{jj}+ n_in_j\partial_j\right]\\
% 	&=6 + 2n_i\partial_i + \partial_i\partial_i-\delta_{jj}-n_j\partial_j-3\delta_{ii}-3n_i\partial_i\\
% 	&\qquad-\delta_{ii}n_j\partial_j-n_i\delta_{ij}\partial_j-n_in_j\partial_i\partial_j \\
% 	&= \partial_i\partial_i -6n_i\partial_i - n_in_j\partial_i\partial_j-6
% \end{align*}


\section{Orientational diffusion in a random flow}

In the introduction of this appendix I argued that the explicit derivation of the Fokker-Planck equation gives an expression for $\mathcal D$. I will briefly illustrate this by computing the diffusion constant for an axisymmetric particle in a random flow. The calculation proceeds exactly as shown in detail above, but here we will use the Jeffery equation for $\ve f(\ve n, t)$:
\begin{align}
	\ve f(\ve n, t) &= \ma O \ve n + \Lambda\left(\ma S \ve n - \ve n \ve n\transpose\ma S \ve n\right), \nn\\
	\ma F(\ve n, t) &= \frac{\rd \ve f}{\rd \ve n} \nn\\
	&= \ma O + \Lambda(\ma S - \maid \ve n\transpose \ma S \ve n - 2\ve n\ve n\transpose \ma S) \nn\\
	& = \ma O + \Lambda( (\maid - 2\ve n \ve n\transpose)\ma S - \maid \ve n\transpose \ma S \ve n )\eqnlab{fjeffery}
\end{align}
Here $\ma A = \ma O + \ma S$ is the matrix of flow gradients.
The relevant gradient statistics of an homogenous, isotropic, and incompressible flow are
\begin{align}
	\left \langle \ma O_{ij}(0)\ma O_{kl}(t)\right \rangle &= -\frac{\tr\langle\ma A\transpose\ma A\rangle}{12}\left(\delta_{ik}\delta_{jl}-\delta_{il}\delta_{jk}\right)\delta(t)\nn\\
	\left \langle \ma S_{ij}(0)\ma S_{kl}(t)\right \rangle &= \frac{\tr\langle\ma A\transpose\ma A\rangle}{60}\left(-2\delta_{ij}\delta_{kl}+3\delta_{ik}\delta_{jl}+3\delta_{il}\delta_{jk}\right)	\delta(t)\nn\\
	\left \langle \ma S_{ij}(0)\ma O_{kl}(t)\right \rangle &= 0 \nn\\
	\left \langle \ma S_{ij}(t)\right \rangle &= 0 \nn\\
	\left \langle \ma O_{ij}(t)\right \rangle &= 0 \eqnlab{appisotropicos}
\end{align}
Inserting \Eqnref{fjeffery} in \Eqnref{deltan}, and taking averages leads to the Fokker-Planck equation coefficients\footnote{In vector notation $(\ve a^{(1)})_i = a^{(1)}_i$ and $(\ma A^{(2)})_i = a^{(2)}_{ij}$}
\begin{align*}
	\ve a^{(1)} &= \frac{1}{2}\left \langle \ma O \ma O \ve n + \Lambda^2\left(
\ma S \ma S \ve n
-2 \ve n \ve n\transpose \ma S\ma S\ve n
-2 (\ve n\transpose \ma S\ve n)\ma S \ve n 
+3 (\ve n\transpose \ma S\ve n)^2 \ve n
\right) \right\rangle\\
\ma A^{(2)}  &= \left\langle -\ma O \ve n \ve n\transpose \ma O + \Lambda^2\left( \ma S \ve n \ve n\transpose\ma S -(\ve n\transpose \ma S \ve n)(\ma S \ve n \ve n\transpose + \ve n \ve n\transpose \ma S - \ve n \ve n\transpose(\ve n\transpose \ma S \ve n))\right) \right\rangle
\end{align*}
And finally, inserting the statistics for a homogenous, isotropic and incompressible flow \Eqnref{appisotropicos} contracts the Fokker-Planck coefficients to 
\begin{align*}
	\ve a^{(1)} &= -\frac{\tr\langle\ma A\transpose \ma A\rangle}{60} \left(5 + 3\Lambda^2 \right)\ve n\\
	\ma A^{(2)}  &= \frac{\tr\langle\ma A\transpose \ma A\rangle}{60} \left(5 + 3\Lambda^2 \right)\left(\maid|\ve n|^2 - \ve n\ve n\transpose \right ).
\end{align*}
By comparison with \Eqnref{a1a2} we conclude that the diffusion constant for a non-spherical particle in a random flow is
\begin{align*}
	\mathcal D &= \frac{\tr\langle\ma A\transpose \ma A\rangle}{120} \left(5 + 3\Lambda^2 \right).
\end{align*}
\end{document}
