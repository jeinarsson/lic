\documentclass[thesis.tex]{subfiles}

\begin{document}

\chapter{Effects of particle and fluid inertia}\seclab{inertia}

This project is a collaboration with Jean-R\'egis Angilella at the University of Caen, in France. Our aim is to understand the effects of inertia on the particle dynamics.

\section{Overview}

The simplest approach to modeling small particles, as measured by the Stokes and particle Reynolds numbers, is to simply let $\st=0$ and $\rep=0$. Clearly there is no real physical system which fulfills this condition strictly. However, it may be very close, as for example in the experiment we describe in Paper A. But it is natural to ask the question what happens when $\st$ and $\rep$ are small, but not strictly equal to zero.

In Paper B we treat the case of $\st>0$, but $\rep=0$. As explained in the paper this case corresponds to particles of larger density than the surrounding fluid. Paper B is self-contained and rather detailed, so I will not repeat any technical detail here. The main results are, first, a modification of the Jeffery equation, that takes into account small values of the Stokes number. Second, we employ the new equation of motion to compute the stationary orientational distribution of weakly inertial particles under the influence of noise. 

The effect of inertia, however small, is particularly interesting for particles in a simple shear flow. When $\st=0$, Jeffery's equation predicts that the orientational trajectory is periodic and always return to the initial orientation. But when $\st>0$, there is instead a drift so that all trajectories, regardless of initial condition, will converge onto the same final orbit. This has previously been shown by Lundell in numerical simulation \cite{lundell2010}. For the case of almost spherical particles, the orbit drift was computed by Subramanian and Koch \cite{subramanian2006}. Our equation of motion confirms and extends these results to arbitrary axisymmetric particle shapes.

We then show how the new equation of motion enables us to compute the orientational distribution of weakly inertial particles under the additional influence of noise. This calculation would be rather difficult without the new equation of motion. The point is that the noise and particle inertia are two different physical effects which independently break the degeneracy of the Jeffery orbits. 

Noise tends to scatter particle orientations, creating a stationary orientational distribution. As explained in \Secref{orientationaldistributions}, for weak noise the orientational distribution converges and becomes independent of the noise strength. But on the other hand, the particle inertia in the absence of noise drives particles into a final orbit. Moreover, the final orbit is different for rod-shaped particles and disk-shaped particles. Therefore we expect a qualitative difference when inertial effects compete with noise. We quantify this competition between inertia and noise in Figs.~4~and~5 in the paper. The results may also be compared to \Figref{sintheta1} in this thesis (p. \pageref{fig:sintheta1}).

Finally, as a technical note, in order to compute the orientational distribution we numerically solve a Fokker-Planck equation on the sphere. The equation is \Eqnref{fpejeff} in \Secref{orientationaldistributions}, but with $\ve J(\ve n, t)$ corresponding to the new equation of motion. The numerical approximation is calculated by rewriting the equation with quantum mechanical angular momentum operators, and expanding the solution in spherical harmonics. I include the details of this calculation as Appendix \ref{app:fpe_sphere} to this thesis, it is also an Appendix of Paper B.

\section{Outlook}

The case of $\st>0$ but $\rep=0$ is only valid for small particles, much heavier than the surrounding fluid, for example aerosols. A natural next step is to consider also $\rep > 0$, which is the relevant case in our experiment. I think it is fair to say that treating a non-zero Stokes number is the simpler task, compared to non-zero particle Reynolds number. A finite $\st$ means that we do perturbation theory on Newton's equations of motion, but a finite $\rep$ entails perturbation theory of the Navier-Stokes equation.

But there exists methods to perform this calculation. Subramanian and Koch \cite{subramanian2005,subramanian2006} calculated the orbit drift due to finite particle and fluid inertia for two special cases: nearly spherical particles and very elongated rods. The curious outcome is that the drift is opposite in the two cases, suggesting that there is a transition somewhere inbetween. An interesting proposition is to compute the existence and stability of periodic orbits, for a general axisymmetric particle. 

\end{document}
