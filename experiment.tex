\documentclass[thesis.tex]{subfiles}

\begin{document}

\chapter[Experimental observations]{Experimental observation of \\single particle orientational dynamics}\label{sec:experiment}

In a collaborative project with the group of Dag Hanstorp (Gothenburg university) we developed an experimental setup for observing the orientational dynamics of single particles. The aim is to understand which mechanisms that are crucial to include in a description of the particle dynamics. By quantitatively measuring the trajectories of individual particles, we aim to narrow down the possible choices of mechanisms driving the particle dynamics. The project has thus far resulted in Paper A of this thesis\footnote{A description of the experimental setup and some preliminary results were published in a conference proceeding not included in this thesis, see Ref.~\cite{mishra2012}.}, which describes observations of both periodic and aperiodic tumbling of rod-shaped particles. At the time of this writing a round of refined experiments are ongoing, and a publication describing the corresponding results is planned.

\section{Overview \& setup}

Since we want to study the orientational dynamics of single particles, our ideal physical system is a fluid flow, with one small, suspended particle. In particular, we would like the following features:
\begin{itemize}
	\item A known fluid flow, which does not change in time as the particle moves.
	\item A known particle geometry, so that the hydrodynamic force can be calculated.
	\item Small particles, in the sense that $\st\ll1$ and $\rep\ll1$.
	\item But not too small particles, in the sense that $\pe\gg1$.
\end{itemize}
We realise this system in a tiny rectangular pipe, a microfluidic channel. Its cross-section is $\unit[0.2]{mm}\times\unit[2.5]{mm}$ and it is manufactured by molding plastic in a precision machined metal mold. The channel is made several centimeters long, and in each end a thin teflon tube is connected as inlet or outlet. A photograph of a channel is shown in Fig.~\ref{fig:exp_setup}. The fluid consists of a mix of glycerol and water, and it is pumped through the channel with a syringe. The channel is mounted in a microscope equipped with a video camera, which in turn is connected to a computer that stores the image sequences.

\begin{figure}
\includegraphics[width=10cm]{figs/expsetup_hires.png}%
\caption{\label{fig:exp_setup} Photograph of a microchannel of the type used in our experiment. The channel is molded in a block of PDMS plastic. In the picture the channel is filled with dye. To the left and right are inlet/outlet tubes. The channel is ca \unit[5]{cm} long.}%
\end{figure}

The particles are plastic rods produced by stretching epoxy glue as it hardens. The rods measure around $\unit[20\textrm{-}40]{\mu m}$ in length and $\unit[1\textrm{-}2]{\mu m}$ in thickness. When we calculate the Stokes and particle Reynolds numbers for a rod in our experiment, we find that $\st \approx 10^{-6}$ and $\rep \approx 10^{-5}$, which qualifies the rod as a small particle. We also compute the rotational Pecl\'et number and find that $\pe\approx10^{4}$. Thus we realise the condition of small, but not too small particles. 

The more complicated features on our wish list are the fluid flow, and the particle geometry. By designing the channel to be wide and shallow we can consider the fluid flow as known. How is discussed in detail in Paper~A. In contrast, the exact shape of the particles remains unknown. As mentioned above, the particles are roughly rod-shaped, with a length ten times longer than its diameter. In a microscope we can easily see features of about $\unit[10]{\mu m}$, but it cannot resolve anything much smaller than $\unit[1]{\mu m}$. Resolving such small scales may seem unnecessary, but can in fact make all the difference. The observation of the effects of these small asymmetries is the main point of Paper~A. In the next section I will extend the discussion of this point a bit further than what is published in the paper.

\section{Results \& discussion}\seclab{expresults}

In Paper A we suggest, without much elaboration, that the data shown in Fig.~8 of the paper correspond to chaotic tumbling of triaxial particles. In this section I want to explain our reasons for this suggestion.

The main results presented in Paper A are the trajectories of $\ve n$ shown in Figs.~6 through 8 in the paper. Recall that $\ve n$ is a unit vector pointing along the axis of symmetry of the rod. From the recorded image sequences we extract $\ve n$ as function of time\footnote{Actually, we extract $\ve n(d)$, where $d$ is the distance advected along the streamline. As shown in the paper this corresponds to time $t$ when writing down for instance Jeffery's equation. This technical detail is of no importance to the present discussion.}. For an axisymmetric rod, the two degrees of freedom contained in $\ve n$ fully determines the orientation of the rod. A third degree of freedom not contained in $\ve n$ is the rotation around the symmetry axis of the rod. Clearly, also an axisymmetric rod may rotate around its symmetry axis, but this \emph{rolling rotation} makes no physical difference. But in the case of an asymmetric rod the rolling is physically relevant, and a description of the orientational dynamics must contain all three rotational degrees of freedom. In our experiment we can not resolve the rolling rotation, so in order to analyse the effects of asymmetry we have to figure out what its signature is in the observable $\ve n$. In the following paragraphs I explain how the signature shows up in the trajectories of $n_z$, shown in panels (c) in all figures in Paper A. 

For the purpose of this discussion we treat the rod as an ellipsoid. An ellipsoid has three perpendicular axes, and the vector $\ve n$ is attached to one of them. We introduce the third degree of freedom by an additional unit vector $\ve p$, attached to another of the ellipsoid axes, perpendicular to $\ve n$. The geometry of the ellipsoid is characterised by two aspect ratios. The aspect ratio $\lambda$ is associated with the axis in the direction of $\ve n$, like in the paper. We denote the second aspect ratio $\kappa$, it is associated with the axis in the direction of $\ve p$. 

In the case of axisymmetric particles, the Jeffery equation of motion for $\ve n$ is derived from a torque balance. In the same way we derive an equation of motion for $\ve n$ and $\ve p$ in the case of a triaxial ellipsoid. The details of this calculation are found in Appendix~\secref{triaxial_equation}. The equations of motion are
\begin{align}
	\diff{\ve{n}}{t} &= \ma O \ve{n} + \frac{\lambda^2-1}{\lambda^2+1} \left(\ma S \ve n- \ve{n}\transpose \ma S \ve{n})\ve{n}\right) + \frac{ 2\lambda^2 (1 - \kappa^2) }{(\lambda^2+\kappa^2)(\lambda^2+1)}(\ve{n}\transpose \ma S \ve{p})\ve{p}, \nn\\
	\diff{\ve{p}}{t} &= \ma O \ve{p} + \frac{\kappa^2-1}{\kappa^2+1}\left(\ma S \ve p - \ve{p}\transpose \ma S \ve{p})\ve{p}\right) + \frac{ 2\kappa^2 (1 - \lambda^2) }{(\kappa^2+\lambda^2)(\kappa^2+1)}(\ve{n}\transpose \ma S \ve{p})\ve{n}.\eqnlab{triaxial_equation}
\end{align}
Here, as in the introduction, $\ma S$ and $\ma O$ are the symmetric and anti-symmetric parts of the flow gradient,
\begin{align*}
	&\ma O = \frac{1}{2}(\ma A - \ma A\transpose),\quad
	\ma S = \frac{1}{2}(\ma A + \ma A\transpose),\quad
	\ma A = \nabla \ve u = \ma O + \ma S.
\end{align*}
We note the similarity between the equations for $\ve n$ and $\ve p$ in \Eqnref{triaxial_equation}. Exchanging the aspect ratios $\lambda \leftrightarrow \kappa$ and the meaning of $\ve n \leftrightarrow \ve p$, must result in the same equations -- it is the same particle. Secondly, we see that the complicated coupling between $\ve n$ and $\ve p$ occurs through the strain $\ma S$. The vorticity $\ma O$ generates a simple solid-body rotation.

In absence of any constraint, the action of the strain matrix $\ma S$ is to collapse both $\ve n$ and $\ve p$ onto its primary eigendirection. But the rigidity of the particle prevents this, and instead a rotation is induced. Exactly which rotation is determined by the particle shape: the strain has more influence on an elongated axis. This is especially clear in the special case where the axis in the direction of $\ve p$ is not elongated at all, that is the axisymmetric case $\kappa = 1$. Then
\begin{align}
	\diff{\ve{n}}{t} &= \ma O \ve{n} + \frac{\lambda^2-1}{\lambda^2+1} \left(\ma S \ve n- \ve{n}\transpose \ma S \ve{n})\ve{n}\right) \nn\\
	\diff{\ve{p}}{t} &= \ma O \ve{p} - \frac{   \lambda^2-1 }{\lambda^2+1}(\ve{n}\transpose \ma S \ve{p})\ve{n}.\eqnlab{triaxial_equation_symm}
\end{align}
In this case we recognise the Jeffery equation for $\ve n$, without any coupling to $\ve p$. But the equation for $\ve p$ is coupled to $\ve n$.  In addition to the simple rotation of the vorticity, the action of the strain on $\ve n$ dictates the motion of $\ve p$ through the rigidity condition. As soon as an asymmetry, that is $\kappa \neq 1$, is introduced, our observable quantity $\ve n$ will be influenced by $\ve p$.

We now move on to discuss the solutions of \Eqnref{triaxial_equation} in simple shear flows. For the axisymmetric case, \Eqnref{triaxial_equation_symm} admits analytical solutions for $\ve n$. The solutions, described in \Secref{jefferyequation}, are the periodic Jeffery orbits. Two important characteristics of the Jeffery orbits are
\begin{enumerate}
	\item they describe monotonous tumbling, the vector $\ve n$ always rotates around the vorticity vector with positive angular velocity,
	\item the orbits are periodic, the particle always returns to its initial condition after one period time.
\end{enumerate}
The general triaxial case does not allow any closed form solution, and generally the solutions are not periodic. Depending on initial condition the solution may be periodic, quasi-periodic or chaotic \cite{hinch1979,yarin1997}. However, as discovered by Hinch and Leal \cite{hinch1979}, the property of monotonous tumbling still holds. That is, the vector $\ve n$ will rotate around the vorticity vector with a positive angular velocity. This allows us to reduce the dimensionality of the problem with a Poincar\'e surface-of-section. In practise this means that we only look at the trajectories as they pass through a plane of our choice. If we choose a plane containing the vorticity vector $\ve \Omega$, the monotonous tumbling guarantees that the trajectory always returns to the surface-of-section.

We choose as surface-of-section the plane $n_x=0$, which is when the rod is perpendicular to the flow direction. In the surface-of-section we choose the coordinates $(\psi, n_z)$. 

The coordinate $n_z$ is the cosine of the angle between $\ve n$ and $\ve \Omega$. It is the same $n_z$ shown in the results of the experiment. For an axisymmetric rod $n_z$ is interpreted as the Jeffery orbit. In the time series $n_z(t)$, the moments where $n_x=0$ correspond exactly to the peak height of the oscillation. The $n_z$-coordinate in the surface-of-section is therefore a quantity easily read from the experimentally observed time series. 

The coordinate $\psi$ is the angle of rolling rotation. As explained above the rolling rotation is not an observable in our experiment. 

By numerical solution of \Eqnref{triaxial_equation} we construct a plot of the Poincar\'e map by choosing an initial condition and computing a long time series. At each crossing of the surface-of-section, we plot a point at $(\psi, n_z)$.

The Poincar\'e map for an axisymmetric rod with aspect ratio $\lambda=10$ and $\kappa=1$ is shown in Fig.~\ref{fig:poincareA}. It consists of only horizontal lines, each corresponding to a constant value of $n_z$. In other words each line corresponds to a different Jeffery orbit. Each line is actually a torus (or a circle) since the angle $\psi$ is periodic. The dynamics is confined to stay on the torus on which it begins. Also shown in Fig.~\ref{fig:poincareA} are examples of trajectories. The successive markers on the surface-of-section correspond in the lower panel to what the experimentally observed time series $n_z(t)$ would look like. 

In general the particle does \emph{not} return to exactly its initial position in $(\psi, n_z)$-space. The dynamics proceeds in jumps of $\Delta \psi$, and the trajectory is periodic if and only if $\pi/\Delta \psi$ is rational. Nevertheless, if we disregard the rolling rotation, the dynamics does appear periodic in $n_z$.

\begin{figure}
\includegraphics[width=12cm]{figs/poincareA.png}%
\caption{\label{fig:poincareA} Top: Poincar\'e surface-of-section of an axisymmetric particle with aspect ratios $\lambda=10$, $\kappa=1$. Bottom: Examples of what $n_z(t)$ looks like, given the trajectory indicated by the color coded markers on the surface-of-section. Surface-of-section image made by Anton Johansson, reproduced under a CC-BY license.}%
\end{figure}
\begin{figure}
\includegraphics[width=12cm]{figs/poincareB.png}%
\caption{\label{fig:poincareB} Top: Poincar\'e surface-of-section of an asymmetric particle with aspect ratios $\lambda=10$, $\kappa=1.1$. Bottom: Examples of what $n_z(t)$ looks like, given the trajectory indicated by the color coded markers on the surface-of-section. Surface-of-section image made by Anton Johansson, reproduced under a CC-BY license.}%
\end{figure}
\begin{figure}
\includegraphics[width=12cm]{figs/poincareC.png}%
\caption{\label{fig:poincareC} Top: Poincar\'e surface-of-section of an axisymmetric particle with aspect ratios $\lambda=10$, $\kappa=1.3$. Bottom: Examples of what $n_z(t)$ looks like, given the trajectory indicated by the color coded markers on the surface-of-section. Surface-of-section image made by Anton Johansson, reproduced under a CC-BY license.}%
\end{figure}

In Fig.~\ref{fig:poincareB} we show the corresponding Poincar\'e map for a slightly asymmetric rod. Here $\lambda=10$ as before, but $\kappa = 1.1$, that is a \unit[10]{\%} asymmetry. In the experiment this small amount of asymmetry corresponds to about $\unit[200]{nm}$ asymmetry in the particle cross-section. The effect of the asymmetry is to bend or destroy the tori. Most tori survive, but in a deformed state. Others are are broken up into a chaotic region, seen as a continuum of unordered dots in the surface-of-section. When the dynamics begins on a torus, it confined to that torus, like before. But if started in the chaotic region, it may, and will eventually, visit all parts of the chaotic region.

The picture is further distorted when we turn the asymmetry up to $\unit[20]{\%}$, as shown in Fig.~\ref{fig:poincareC}. Here the chaotic region has grown, and more tori have broken up. But it is still true that trajectories which started on a torus, are confined to that torus. In the examples of $n_z(t)$ given in Figs.~\ref{fig:poincareA}-\ref{fig:poincareC} the dynamics on the torus shows up as a quasi-periodic variation. The trajectory started in the chaotic region results in a chaotic variation in $n_z(t)$.

In Paper A it is rather briefly suggested (p. 12) that the data corresponds to chaotic tumbling of a triaxial particle. I have here explained the reasons for making this suggestion. The data presented in Paper A is consistent with the trajectories in the surface-of-section, given an asymmetry of \unit[10-30]{\%}. We see regular motion, corresponding to the tori, but also chaotic motion. The tori close to $n_z=0$ are the first to break up, and are very rarely observed, while tori close to $n_z=1$ are more common. 


\section{Outlook}

A complicating factor in our analysis is that each experimentally recorded trajectory corresponds to a different particle, and therefore to a different surface-of-section. Removing this problem was a main aim in the design the current iteration of the experiment.

The idea is to use a laser, an optical tweezer, to control the initial condition $n_z$ of a rod. After observing the trajectory of a tumbling particle, the laser is used to capture the very same particle again. The particle is brought back to its starting position and reset with a new initial condition. By repeating this procedure we want to map out which tori are bent, and which are unaffected. This work is underway by M.Sc. students Staffan Ankardal \& Alexander Laas.

Another important question is to consider the alternative mechanisms that could affect the particle dynamics. The effect of thermal noise has been studied by many authors for the case of axisymmetric particles. A little bit of noise invalidates the rule that a trajectory is confined to its tori, in the case of axisymmetric particle this means a Jeffery orbit. Instead there is an orientational distribution over different tori.

But the corresponding question arises also for non-axisymmetric particles. A little bit of noise allows for transport between tori, and we expect an orientational distribution to exist.

Finally, for larger triaxial particles both fluid and particle inertia contribute to the dynamics. In Paper~B (see \Secref{inertia}) we compute the first effects of particle inertia ($\st>0$) on the dynamics of an axisymmetric particle. The corresponding calculation for a triaxial particle remains as an open question. Fluid inertia ($\rep > 0$) is more difficult, even in the case of axisymmetric particles (see \Secref{inertiaoutlook}).



\end{document}
